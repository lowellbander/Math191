\documentclass[11pt]{article}

% \setlength{\topmargin}{9pt}
% \setlength{\textheight}{9in}
% \setlength{\headheight}{2pt}
% \setlength{\headsep}{4pt}
% \setlength{\oddsidemargin}{0.25in}
% \setlength{\textwidth}{6in}
% \pagestyle{plain}


\usepackage{amsmath,amssymb,amsrefs}
\usepackage{amscd}
\usepackage{latexsym}
\usepackage{graphics}
\usepackage{amssymb}


\setlength{\oddsidemargin}{0in}
\setlength{\evensidemargin}{0in}
\addtolength{\topmargin}{-1in}
\setlength{\textwidth}{6.5in}
\setlength{\textheight}{8in}

\usepackage{algorithm}
\usepackage{algorithmic}

% \setlength{\topmargin}{0in}
% \setlength{\headheight}{0in}
% \setlength{\headsep}{0in}
% \setlength{\textheight}{7.7in}
% \setlength{\textwidth}{6.5in}
% \setlength{\oddsidemargin}{0in}
% \setlength{\evensidemargin}{0in}
% \setlength{\parindent}{0.25in}
% \setlength{\parskip}{0.25in}

\usepackage{subfigure}
\usepackage{graphicx}

%
% this command enables to remove a whole part of the text 
% from the printout
% to use it just enter
% \remove{  
% before the text to be excluded and
% } 
% after the text
\newcommand{\remove}[1]{}

%
% The following macros are used to generate nice code for programs.
% See example on how to use it below
%

%%%%%%%%%%%%%%%%%%%%% program macros %%%%%%%%%%%%%%%%%


%%%%%%%%%%%%%%%%%%%%% End of PROGRAM macros %%%%%%%%%%%%%%%%%



\newcommand{\lecture}[5]{
   \pagestyle{headings}
   \thispagestyle{plain}
   \newpage
%   \setcounter{chapter}{#1}
%   \setcounter{page}{#2}
%  \set\thechapter{#3}
   \noindent
   \begin{center}
   \framebox{
      \vbox{
    \hbox to 6.28in { {\bf MATH 191 Topics in Data Science: Algorithms and Math. Foundations
                        \hfill  - Fall 2015} }
       \vspace{4mm}
       \hbox to 6.28in { {\Large \hfill Lecture #1: #3  \hfill} }
       \vspace{2mm}
       \hbox to 6.28in { {\it Lecturer: #4 \hfill Scribes: #5} }
      }
   }
   \end{center}
   \markboth{Lecture #1: #3}{Lecture #1: #3}
   \vspace*{4mm}
}

%
% Use these macros for organizing sections of your notes.
% Each command takes two arguments: (1) the title of the section and and
% (2) a keyword for that section to appear in the index.  (See examples.)
% Please don't use \section, \subsection, and \subsubsection directly!
%

\newcommand{\topic}[2]{\section{#1} \index{#2} \markright{#1}}
\newcommand{\subtopic}[2]{\subsection{#1} \index{#2}}
\newcommand{\subsubtopic}[2]{\subsubsection{#1} \index{#2}}
 
%
% Convention for citations is first author's last name followed by other
% authors' last initials, followed by the year.  For example, to cite the
% seventh entry in the course bibliography, you would type: \cite{BurnsL80}
% (To avoid bibliography problems, for now we redefine the \cite command.)
%

\renewcommand{\cite}[1]{[#1]}

%
% These are just to make things a little easier:
%
\newcommand{\bi}{\begin{itemize}}
\newcommand{\ei}{\end{itemize}}
\newcommand{\be}{\begin{enumerate}}
\newcommand{\ee}{\end{enumerate}}
\newcommand{\blank}{\vspace{1ex}}   % generates a blank line in the output

%
% Use these for theorems, lemmas, proofs, etc.
%
\newtheorem{theorem}{Theorem}
\newtheorem{lemma}[theorem]{Lemma}
\newtheorem{claim}[theorem]{Claim}
\newtheorem{corollary}[theorem]{Corollary}
\newcommand{\qed}{\hfill $\Box$}
% \newenvironment{proof}{\par{\bf Proof:}}{\qed \par}
\newenvironment{proof}{{\em Proof:}}{\hfill\rule{2mm}{2mm}}

%
% Use the following for definitions.
% \bigdef is for definitions to be set off by themselves; \smalldef is for
% definitions given in the middle of a paragraph.
%
\newenvironment{dfn}{{\vspace*{1ex} \noindent \bf Definition }}{\vspace*{1ex}}
\newcommand{\bigdef}[2]{\index{#1}\begin{dfn} {\rm #2} \end{dfn}}
\newcommand{\smalldef}[1]{\index{#1} {\em #1}}
% **** IF YOU WANT TO DEFINE ADDITIONAL MACROS FOR YOURSELF, PUT THEM HERE:
% \usepackage{subfigure}
% \usepackage{graphicx}

\begin{document}

\lecture{4}{1}{October 2, 2015}{Mihai Cucuringu}{Eric Aberbook \& Lowell Bander}

\section{Bias-Variance Decomposition}

If we let $z$ be a random variable, then we can express the variance of $z$ as follows.

\begin{align}
Var(z) &= E[(z-E(z))^2]\\
&= E[z^2 - 2zE(z) + (E(z))^2]\\
&= E(z^2) - 2E(zE(z)) + E[E(z)^2]\\
&= E(z^2) - 2(E(z))^2 + (E(z))^2\\
&= E(z^2) - (E(z))^2
\end{align}

\begin{lemma}
\begin{align}
E(z^2) = E[(z-E(z)^2)]+[E(z)^2]
\end{align}
\end{lemma}

\begin{proof}


Not yet sure how things fit together, so just gonna transcribe the equations I have for now.\\

\begin{align}
E(z^2) &= E[(z - E(z))^2]
\end{align}

\begin{align}
y = f(x) + \epsilon : \epsilon \sim N(0, \Delta^2) % very low confidence on the second parameter here.
\end{align}

Let $\{(x_i, y_i) | i = 1, N\}$ be the training sample. We want to fit some hypothesis function $h(x)$ [e.g. $h(x) = ax + b$] such that we minimize the squared error represented by $\sum_{i=1}^{N} (y_i - h(x_i))^2$. \\

Imagine a new test points coming in represented by $(x_o, y_o)$ such that they are distributed by a probability density function $p$.\\

Observe that $y_o = f(x_o) + \epsilon$.\\

Our goal is to minimize reducible error, so let us express the reducible error in terms of the bias error and the variance error. If we let $y_o$ be the ground truth and $h(x_o)$ be our forecast, then the reducible error can be expressed as

\begin{align}
E[(y_o - h(x_o))^2] &= E[y_o^2 - 2h(x_o)y_o + (h(x_o))^2]\\
&=E[y_o^2] - 2E[h(x_o)\cdot y_o] + E[(h(x_o))^2]\\
&= E[(y_o - E(y_o))^2] + (E(y_o))^2 - 2\cdot E[h(x_o)\cdot (f(x_o) + \epsilon)] + E[(h(x_o) - \hbar(x_o))^2] + (\hbar(x_o))^2
\end{align}

Since $\epsilon$ has a mean of zero, the term $2\cdot E[h(x_o)\cdot (f(x_o) + \epsilon)]$ can be rewritten as $2E[h(x_o)\cdot f(x_o)]$ which can in turn be rewritten as $2f(x_o)\cdot\hbar(x_o)$. And so, our above decomposition continues as 

\begin{align}
&= E[(y_o - E(y_o))^2] + (E(y_o))^2 - 2f(x_o)\cdot\hbar(x_o) + E[(h(x_o) - \hbar(x_o))^2] + (\hbar(x_o))^2\\
&= E[(y_o - E(y_o))^2] + (E(y_o))^2 - 2f(x_o)\cdot\hbar(x_o) + E[(h(x_o)+E[(h(x_o) - \hbar(x_o))])^2] + (\hbar(x_o))^2\\
&= E(\epsilon^2) + E[(h(x_o) - \hbar(x_o))^2] + (f(x_o))^2 - 2f(x_o)\hbar(x_o) + (\hbar(x_o))^2\\
&= E(\epsilon^2) + E[(h(x_o) - \hbar(x_o))^2] + (f(x_o) - \hbar(x_o))^2\\
&=Var(\epsilon) + Var(h) + Bias(h(x_o))
\end{align}

\end{proof}

\section{Pearson Correlation Measures}

The Pearson correlation, represented by \(\rho\) is the measure of linear dependence between variables.
There are two representations of \(\rho\): \\

Population Correlation:
\begin{align}
Cor(X,Y)
&= \dfrac{Cov(X,Y)}{\sqrt{Var(X)}\sqrt{Var(Y)}}
\end{align}

Sample Correlation:
\begin{align}
cor(x,y)
&= \dfrac{cov(x,y)}{\sqrt{var(X)}\sqrt{var(Y)}} \\
&= \dfrac{(x - \bar{x}1)(y - \bar{y}1)}{\| \mathbf{x - \bar{x}1} \|\| \mathbf{y - \bar{y}1} \|}
\end{align}

Note the difference between Population and Sample- \\
Sample- A subset of items from a larger population that is collected to make inferences \\
Population- The total collection of items about which you want to make inferences.



\section{Properties of Population Correlation}
The following are all relevant properties of \(\rho\):

\begin{itemize}
\item $Cor(X,X) = 1$
\item $Cor(X,Y) = Cor(Y,X)$
\item \textit{Cor(aX + b, Y)} = sign(a)Cor(X,Y) for any a,b $\epsilon$ R
\item $-1 \leq \textit{Cor(X,Y)}  \leq 1 $
\item $|Cor(X,Y)  =  1|$ if and only if $Y = aX + b$ for some a,b $\epsilon \mathbf{R}$ with a $\neq$ 0 
\item If  \textit{(X,Y)} are independent then \textit{Cor(X,Y)} = 0
\item If $Cor(X,Y) = 0$ then $X,Y$ need not be independent.
%\item If $(X,Y)$ is bivariate normal and $Cor(X,Y) = 0$, then $X,Y$ are independent


\end{itemize}

\section{Bivariate Normal Distribution}
Two-dimensional Gaussian distribution
The random vector $Z = (X,Y)$ $\epsilon$ $\mathbf{R}$ has a bivariate normal distribution


\section{Properties of Sample Correlation}

\begin{itemize}
\item $cor(x,x) = 1$
\item $cor(x,y) = cor(y,x)$
\item \textit{cor(ax + b, y)} = sign(a)cor(x,y) for any a,b $\epsilon$ R
\item $-1 \leq \textit{cor(x,y)}  \leq 1 $
\item $|cor(x,y)  =  1|$ if and only if $y = ax + b$ for some a,b $\epsilon \mathbf{R}$ with a $\neq$ 0
\item $cor(x,y) = 0$ iff x,y are orthogonal
\item If $x,y$ are centered then $cor(x,y) = cos\theta$, where $\theta$ is the angle between the vectors $x,y$ $\epsilon \mathbf{R^n}$ 



\end{itemize}

\section{Drawbacks of Pearson Correlation}
The Pearson Coefficient $\rho$ has many applications; it is by far the most popular tool in the process of understanding bivariate relationships, and it also has easy means of calculation and interpretability.

Some of the drawbacks are as follows:

\begin{itemize}
\item $\rho$ does not guarantee a casual relationship between variables
\item just because there is no correlation between variables, we can not conclude there is no relationship between them
\item it is only suited well to continuous, normally distributed data
\item $\rho$ is easily corrupted by outliers
\item it is only a measure of \textbf{linear} dependency; does not work for the vast amount of nonlinear data out there

\end{itemize}


\end{document}